\lhead{1\up{ère} NSI}
\chead{\Large \textbf{1. De Scratch à Python}}
\rhead{Python}
\lfoot{Lycée Ismaël Dauphin -- Cavaillon}
\cfoot{}
\rfoot{page \thepage{} sur \pageref{fin}}
\renewcommand{\headrulewidth}{1pt}
\renewcommand{\footrulewidth}{1pt}

Remarque : le bloc \begin{scratch}[baseline=1]\blockinit{quand \greenflag est cliqué}\end{scratch} ne se traduit pas en python.

\begin{tabularx}{\linewidth}{|*{2}{X|}}\hline
    \centering \arraybackslash \textbf{Scratch} & \centering \arraybackslash \textbf{Python} \\\hline
    
    % Afficher un texte
    \multicolumn{2}{|c|}{\textbf{Afficher un texte}}\\\hline
    \begin{scratch}
        \blocklook{dire \ovalnum{Bonjour}}
    \end{scratch}
    &\begin{minipage}{\textwidth}
        \lstinputlisting{print.py}
    \end{minipage}
    \\\hline
    
    % Assigner une valeur à une variable (nombre + texte)
    \multicolumn{2}{|c|}{\textbf{Donner une valeur à une variable}}\\\hline
    \begin{scratch}
        \blockvariable{mettre \selectmenu{prenom} à \ovalnum{Anna}}
    \end{scratch}
    &\begin{minipage}{\textwidth}
        \lstinputlisting{prenom.py}
    \end{minipage}
    \\\hline
    \begin{scratch}
        \blockvariable{mettre \selectmenu{age} à \ovalnum{18}}
    \end{scratch}
    &\begin{minipage}{\textwidth}
        \lstinputlisting{age.py}
    \end{minipage}
    \\\hline
    
    % Afficher la valeur d'une variable
    \multicolumn{2}{|c|}{\textbf{Afficher la valeur d'une variable}}\\\hline
    \begin{scratch}
        \blocklook{dire \ovalvariable{prenom}}
    \end{scratch}
    &\begin{minipage}{\textwidth}
        \lstinputlisting{print_var.py}
    \end{minipage}
    \\\hline
    
    % Incrémenter une variable
    \multicolumn{2}{|c|}{\textbf{Incrémenter une variable}}\\\hline
    \begin{scratch}
        \blockvariable{ajouter \ovalnum{1} à \selectmenu{age}}
    \end{scratch}
    &\begin{minipage}{\textwidth}
        \lstinputlisting{incrementer.py}
    \end{minipage}
    \\\hline
    
    % Demander une valeur
    \multicolumn{2}{|c|}{\textbf{Demander une valeur à l'utilisateur}}\\\hline
    \multicolumn{2}{|c|}{\textit{Demander un texte}}\\\hline
    \begin{scratch}
        \blocksensing{demander \ovalnum{Ton nom ?} et attendre}
        \blockvariable{mettre \selectmenu{nom} à \ovalsensing{réponse}}
    \end{scratch}
    &\begin{minipage}{\textwidth}
        \lstinputlisting{input.py}
    \end{minipage}
    \\\hline
    \multicolumn{2}{|c|}{\textit{Demander un nombre entier}}\\\hline
    \begin{scratch}
        \blocksensing{demander \ovalnum{Ton age ?} et attendre}
        \blockvariable{mettre \selectmenu{age} à \ovalsensing{réponse}}
    \end{scratch}
    &\begin{minipage}{\textwidth}
        \lstinputlisting{inputage.py}
    \end{minipage}
    \\\hline
    \multicolumn{2}{|c|}{\textit{Demander un nombre décimal}}\\\hline
    \begin{scratch}
        \blocksensing{demander \ovalnum{Taille ?} et attendre}
        \blockvariable{mettre \selectmenu{taille} à \ovalsensing{réponse}}
    \end{scratch}
    &\begin{minipage}{\textwidth}
        \lstinputlisting{inputtaille.py}
    \end{minipage}
    \\\hline
    
    % Concaténer deux textes
    \multicolumn{2}{|c|}{\textbf{Concaténer deux textes}}\\\hline
    \multicolumn{2}{|c|}{\textit{Dans une variable}}\\\hline
    \begin{scratch}
        \blockvariable{mettre \selectmenu{txt} à \ovaloperator{regrouper \ovalnum{Bonjour~} et \ovalvariable{nom}}}
    \end{scratch}
    &\begin{minipage}{\textwidth}
        \lstinputlisting{concat.py}
    \end{minipage}
    \\\hline
    \begin{scratch}
        \blockvariable{mettre \selectmenu{txt} à \ovaloperator{regrouper \ovalvariable{age} et \ovalnum{ans.}}}
    \end{scratch}
    &\begin{minipage}{\textwidth}
        \lstinputlisting{concatage.py}
    \end{minipage}
    \\\hline
    \multicolumn{2}{|c|}{\textit{Pour affichage}}\\\hline
    \begin{scratch}
        \blocklook{dire \ovaloperator{regrouper \ovalnum{Bonjour~} et \ovalvariable{nom}}}
    \end{scratch}
    &\begin{minipage}{\textwidth}
        \lstinputlisting{concatprint.py}
    \end{minipage}
    \\\hline
    \begin{scratch}
        \blocklook{dire \ovaloperator{regrouper \ovaloperator{regrouper \ovalnum{Tu as } et \ovalvariable{age}} et \ovalnum{ ans.}}}
    \end{scratch}
    &\begin{minipage}{\textwidth}
        \lstinputlisting{concatprintage.py}
    \end{minipage}
    \\\hline
    
\end{tabularx}

\begin{tabularx}{\linewidth}{|*{2}{X|}}\hline
    \centering \arraybackslash \textbf{Scratch} & \centering \arraybackslash \textbf{Python} \\\hline
    
    % Boucle pour
    \multicolumn{2}{|c|}{\textbf{Répéter des instructions}}\\\hline
    \begin{scratch}
        \blockrepeat{répéter \ovalnum{10} fois}{\blocklook{dire \ovalnum{Bonjour}}}
    \end{scratch}
    &\begin{minipage}{\textwidth}
        \lstinputlisting{for.py}
    \end{minipage}
    \\\hline
    
    % Si
    \multicolumn{2}{|c|}{\textbf{Condition si}}\\\hline
    \begin{scratch}
        \blockif{si \booloperator{\ovalvariable{heure} < \ovalnum{12}} alors}{\blocklook{dire \ovalnum{matin}}}
    \end{scratch}
    &\begin{minipage}{\textwidth}
        \lstinputlisting{if.py}
    \end{minipage}
    \\\hline
    
    % Si -- Sinon
    \multicolumn{2}{|c|}{\textbf{Condition si - sinon}}\\\hline
    \begin{scratch}
        \blockifelse{si \booloperator{\ovalvariable{heure} < \ovalnum{12}} alors}
            {\blocklook{dire \ovalnum{matin}}}
            {\blocklook{dire \ovalnum{après-midi}}}
    \end{scratch}
    &\begin{minipage}{\textwidth}
        \lstinputlisting{ifelse.py}
    \end{minipage}
    \\\hline
    
    % Si -- Sinon si -- sinon
    \multicolumn{2}{|c|}{\textbf{Condition si - sinon si - sinon}}\\\hline
    \begin{scratch}
        \blockifelse{si \booloperator{\ovalvariable{heure} < \ovalnum{12}} alors}
            {\blocklook{dire \ovalnum{matin}}}
            {\blockifelse{si \booloperator{\ovalvariable{heure} = \ovalnum{12}} alors}
                {\blocklook{dire \ovalnum{midi}}}
                {\blocklook{dire \ovalnum{après-midi}}}}
    \end{scratch}
    &\begin{minipage}{\textwidth}
        \lstinputlisting{ifelifelse.py}
    \end{minipage}
    \\\hline
    
    % Jusqu'à ce que / tant que
    \multicolumn{2}{|c|}{\textbf{Jusqu'à ce que (scratch) / tant que (python)}}\\\hline
    \begin{scratch}
        \blockvariable{mettre \selectmenu{i} à \ovalnum{0}}
        \blockrepeat{répéter jusqu'à ce que \booloperator{\ovalvariable{i} > \ovalnum{9}}}
        {
            \blocklook{dire \ovalvariable{i}}
            \blockvariable{ajouter \ovalnum{1} à \selectmenu{i}}
        }
    \end{scratch}
    &\begin{minipage}{\textwidth}
        \lstinputlisting{while.py}
    \end{minipage}
    \\\hline
    
\end{tabularx}

\newpage

{\huge Exercices}

\large

\begin{exo}~\\
    Sur Scratch, traduire le programme python suivant :
    \lstinputlisting{exo1.py}
\end{exo}

\begin{exo}~\\
    Traduire en python le programme suivant :
    \begin{center}
        \begin{scratch}
            \blockinit{quand \greenflag est cliqué}
            \blocksensing{demander \ovalnum{Choisir un nombre} et attendre}
            \blockvariable{mettre \selectmenu{A} à \ovalsensing{réponse}}
            \blockvariable{mettre \selectmenu{A} à \ovaloperator{\ovalvariable{A} $+$ \ovalnum{3}}}
            \blockvariable{mettre \selectmenu{A} à \ovaloperator{\ovalvariable{A} $*$ \ovalnum{2}}}
            \blockvariable{mettre \selectmenu{A} à \ovaloperator{\ovalvariable{A} $-$ \ovalnum{5}}}
            \blocklook{dire \ovaloperator{regrouper \ovalnum{Le programme de calcul donne} et \ovalvariable{A}}}
        \end{scratch}
    \end{center}
\end{exo}

\begin{exo}~\\
    Traduire en python le programme suivant :
    \begin{center}
        \begin{scratch}
            \blockinit{quand \greenflag est cliqué}
            \blockvariable{mettre \selectmenu{puissance} à \ovalnum{1}}
            \blockrepeat{répéter \ovalnum{10} fois}{
                \blockvariable{mettre \selectmenu{puissance} à \ovaloperator{\ovalvariable{puissance} $*$ \ovalnum{2}}}
            }
            \blocklook{dire \ovalvariable{puissance}}
        \end{scratch}
    \end{center}
\end{exo}

\newpage

\begin{exo}~\\
    Traduire en python le programme suivant :
    \begin{center}
        \begin{scratch}[scale=0.8]
            \blockinit{quand \greenflag est cliqué}
            \blocksensing{demander \ovalnum{De quel nombre souhaitez vous la table ?} et attendre}
            \blockvariable{mettre \selectmenu{a} à \ovalsensing{réponse}}
            \blockvariable{mettre \selectmenu{n} à \ovalnum{0}}
            \blockrepeat{répéter \ovalnum{11} fois}
                {
                    \blocklook{dire \ovaloperator{regrouper \ovalvariable{a} et \ovaloperator{regrouper \ovalnum{x} et \ovaloperator{regrouper \ovalvariable{n} et \ovaloperator{regrouper \ovalnum{=} et \ovaloperator{\ovalvariable{a} $*$ \ovalvariable{n}}}}}}}
                    \blockvariable{ajouter \ovalnum{1} à \selectmenu{n}}
                }
        \end{scratch}
    \end{center}
\end{exo}

\begin{exo}~\\
    Traduire en python le programme suivant :
    \begin{center}
        \begin{scratch}
            \blockinit{quand \greenflag est cliqué}
            \blocksensing{demander \ovalnum{Donner la longueur du côté AB} et attendre}
            \blockvariable{mettre \selectmenu{AB} à \ovalsensing{réponse}}
            \blocksensing{demander \ovalnum{Donner la longueur du côté AC} et attendre}
            \blockvariable{mettre \selectmenu{AC} à \ovalsensing{réponse}}
            \blocksensing{demander \ovalnum{Donner la longueur du côté BC} et attendre}
            \blockvariable{mettre \selectmenu{BC} à \ovalsensing{réponse}}
            \blockifelse{si \booloperator{
                \ovaloperator{
                    \ovaloperator{\ovalvariable{AB} $*$ \ovalvariable{AB}}
                    $+$
                    \ovaloperator{\ovalvariable{BC} $*$ \ovalvariable{BC}}
                } 
                $=$
                \ovaloperator{
                    \ovalvariable{AC} $*$ \ovalvariable{AC}}}
                alors}
                {\blocklook{dire \ovalnum{Le triangle ABC est rectangle en B.}}}
                {\blocklook{dire \ovalnum{Le triangle ABC n'est pas rectangle en B.}}}
        \end{scratch}
    \end{center}
\end{exo}

\newpage

\begin{exo}~\\
    Traduire en python le programme suivant :
    \begin{center}
        \begin{scratch}
            \blockinit{quand \greenflag est cliqué}
            \blockvariable{mettre \selectmenu{i} à \ovalnum{0}}
            \blockrepeat{répéter jusqu'à ce que \booloperator{\ovaloperator{\ovalvariable{i} $*$ \ovalvariable{i}} $>$ \ovalvariable{3333}}}{
                \blockvariable{ajouter \ovalnum{1} à \selectmenu{i}}
            }
            \blocklook{dire \ovalvariable{i}}
        \end{scratch}
    \end{center}
\end{exo}

\begin{exo}~\\
    Traduire en python le programme suivant :
    \begin{center}
        \begin{scratch}
            \blockinit{quand \greenflag est cliqué}
            \blocksensing{demander \ovalnum{Quelle est la valeur de a ?} et attendre}
            \blockvariable{mettre \selectmenu{a} à \ovalsensing{réponse}}
            \blocksensing{demander \ovalnum{Quelle est la valeur de b ?} et attendre}
            \blockvariable{mettre \selectmenu{b} à \ovalsensing{réponse}}
            \blockrepeat{répéter jusqu'à ce que \booloperator{\ovalvariable{a} $=$ \ovalvariable{b}}}
            {
            \blockifelse{si \booloperator{\ovalvariable{a} > \ovalvariable{b}} alors}
                {
                    \blockvariable{mettre \selectmenu{a} à \ovaloperator{\ovalvariable{a} $-$ \ovalvariable{b}}}
                }
                {
                    \blockvariable{mettre \selectmenu{b} à \ovaloperator{\ovalvariable{b} $-$ \ovalvariable{a}}}
                }
            }
            \blocklook{dire \ovaloperator{regrouper \ovalnum{Le PGCD de a et b est } et \ovalvariable{a}}}
        \end{scratch}
    \end{center}
\end{exo}

\newpage

\begin{exo}~\\
    Traduire en python le programme suivant :
    \begin{center}
        \begin{scratch}
            \blockinit{quand \greenflag est cliqué}
            \blocksensing{demander \ovalnum{Choisir un premier nombre} et attendre}
            \blockvariable{mettre \selectmenu{a} à \ovalsensing{réponse}}
            \blocksensing{demander \ovalnum{Choisir un deuxième nombre} et attendre}
            \blockvariable{mettre \selectmenu{b} à \ovalsensing{réponse}}
            \blocksensing{demander \ovalnum{Choisir un troisième nombre} et attendre}
            \blockvariable{mettre \selectmenu{c} à \ovalsensing{réponse}}
            \blockifelse{si \booloperator{\ovalvariable{a} > \ovalvariable{b}} alors}
                {
                \blockifelse{si \booloperator{\ovalvariable{a} > \ovalvariable{c}} alors}
                    {\blocklook{dire \ovaloperator{regrouper \ovalnum{le plus grand des trois nombres est} et \ovalvariable{a}}}}
                    {\blocklook{dire \ovaloperator{regrouper \ovalnum{le plus grand des trois nombres est} et \ovalvariable{c}}}}
                }
                {
                \blockifelse{si \booloperator{\ovalvariable{b} > \ovalvariable{c}} alors}
                    {\blocklook{dire \ovaloperator{regrouper \ovalnum{le plus grand des trois nombres est} et \ovalvariable{b}}}}
                    {\blocklook{dire \ovaloperator{regrouper \ovalnum{le plus grand des trois nombres est} et \ovalvariable{c}}}}
                }
        \end{scratch}
    \end{center}
\end{exo}

\label{fin}
